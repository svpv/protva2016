\documentclass[russian,a4paper,12pt]{article}
\usepackage[utf8]{inputenc}
\usepackage[russian]{babel}
\usepackage{url}
\usepackage{fullpage}

\sloppy
\hyphenpenalty=666

\begin{document}
\title{Типизация ABI}
\author{Алексей Турбин}
\date{21 сентября 2016 г.}
\maketitle

\begin{abstract}
Имеющаяся реализация бинарных зависимостей на основе set-версий
позволяет контролировать наличие нужных символов в библиотеках.
Показано, что реализация может быть расширена таким образом,
чтобы учитывался и тип символа, а не только его имя; так чтобы
обеспечивалось не только наличие нужных функций, но и совместимость
их прототипов.
\end{abstract}

\section{Обзор и введение}
В предыдущей работе [1] показано, что понятие \textit{обратной совместимости}
часто является обманчивым; для достижения реальной \textit{бинарной
совместимости} между библиотеками и программами (ABI) нужно контролировать
не только и не столько версии библиотек, сколько \textit{разрешимость символов}
(т.\,е. наличие нужных функций в библиотеках).  Там же обсуждается, насколько
компактным может быть вероятностное представление множеств, при котором каждый
символ представлен $n$-битным хешем.

\section{Типизация}
\section{Сопоставление}
\section{Дополнительная типизация}

\begin{thebibliography}{9}

\bibitem{2}
Ulrich Drepper. \textit{How To Write Shared Libraries.}\\
\url{https://www.akkadia.org/drepper/dsohowto.pdf}

\end{thebibliography}

\end{document}
